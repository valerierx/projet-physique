\documentclass[12pt,a4paper]{article}
\usepackage[french]{babel}
\usepackage{amsmath}
\usepackage{amsfonts}
\usepackage{amssymb}
\usepackage{titling}
\usepackage[margin=2.5cm]{geometry}
\usepackage[T1]{fontenc}
\usepackage{graphicx} % Required for inserting images

\date{Juin 2025}

\newcommand{\subtitle}[1]{%
  \posttitle{%
    \par\end{center}
    \begin{center}\LARGE#1\end{center}
    \vskip0.5em}%
}

\title{\textbf{CY Tech
\\{\Large Projet numérique}}}
\subtitle{Justification de l'algorithme}
\author{Yanis KASSOU, Gweltaz COLLIN et Valérie ROUX}

\begin{document}

\maketitle

On cherche une solution particulière de:

\[\psi(x,t) = \phi(x)e^{-\frac{iEt}{\hbar}}\]

En injectant dans l'équation de Schrödinger:

\[i\hbar\left(-\frac{iE}{\hbar}\right)\phi(x)e^{-\frac{iEt}{\hbar}} = H\phi(x)e^{-\frac{iEt}{\hbar}}\]

\[\Rightarrow E\phi(x) = H\phi(x)\]
\[\Rightarrow E\phi(x) = -\frac{\hbar^2}{2m}\frac{d^2\phi}{dx^2} + V(x)\phi(x)\]


$\phi(x)$ est vecteur propre donc un état propre de l'énergie,
et $E$ est valeur propre (l'énergie associée).

H opérateur ou matrice

On pose le développement de Taylor autour de $x_i$:
\[\phi(x_i+dx) = \phi(x_i) + dx\,\phi'(x_i) + \frac{dx^2}{2}\phi''(x_i)\]
\[\phi(x_i-dx) = \phi(x_i) - dx\,\phi'(x_i) + \frac{dx^2}{2}\phi''(x_i)\]

Somme des deux:
\[\phi(x_i+dx) + \phi(x_i-dx) = 2\phi(x_i) + dx^2\phi''(x_i)\]

\[\phi''(x_i) = \frac{\phi(x_i+dx)-2\phi(x_i)+\phi(x_i-dx)}{dx^2}\]

On approxime $\frac{d^2\phi}{dx^2}$ par $\frac{\phi_{i+1}-2\phi_i+\phi_{i-1}}{dx^2}$
\[H = -\frac{\hbar^2}{2m}\frac{d^2}{dx^2} + V(x)\]

On obtient:

\[E\phi(x) = -\frac{\hbar}{2m}\cdot\frac{\phi(x_i+dx)-2\phi(x_i)+\phi(x_i-dx)}{dx^2}+V(x)\phi(x)\]

\[\Rightarrow -E\phi(x_i) = -\frac{\hbar^2}{2mdx^2}\phi(x_{i+1})+(\frac{\hbar^2}{mdx^2}+V_i)\phi(x_i)-\frac{\hbar^2}{2mdx^2}\phi(x_{i-1})\]

Matrice diagonale principale
\[H_{ii} = \frac{\hbar^2}{mdx^2}+V(x_i)\]


Matrice diagonale inférieure et supérieure
\[H_{i,i\pm1} = -\frac{\hbar^2}{2mdx^2}\]

\underline{Remarque}: dans le code Python, on a mis $\hbar$ et m égal à 1 pour simplifier.
\end{document}