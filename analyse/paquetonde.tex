\documentclass[12pt,a4paper]{article}
\usepackage[french]{babel}
\usepackage{amsmath}
\usepackage{amsfonts}
\usepackage{amssymb}
\usepackage{titling}
\usepackage[margin=2.5cm]{geometry}
\usepackage[T1]{fontenc}
\usepackage{graphicx} % Required for inserting images

\date{Juin 2025}

\newcommand{\subtitle}[1]{%
  \posttitle{%
    \par\end{center}
    \begin{center}\LARGE#1\end{center}
    \vskip0.5em}%
}

\title{\textbf{CY Tech
\\{\Large Projet numérique}}}
\subtitle{Étude analytique du paquet d'onde}
\author{Yanis KASSOU, Gweltaz COLLIN et Valérie ROUX}

\begin{document}

\maketitle

\section{Méthode stationnaire (Onde plane)}

$\Psi(x) = \Psi_0 e^{i\vec{k}x}$ avec $\vec{k}$: vecteur d'onde

\[E_n = \frac{1}{2m}\left(\frac{n\pi\hbar}{a}\right)^2 - V_0\]

Dans cette approche, la particule est modélisée par une onde plane infinie, ce qui revient à la considérer comme délocalisée dans l'espace. Elle possède une énergie unique, comme démontrée précédemment par résolution de l'équation de Schrödinger indépendante du temps pour déterminer le coefficient de transmission, l'énergie...

Ainsi, il s'agit donc d'une description statique sans notion du temps.

Cette méthode ne décrit ni le mouvement ni l'évolution temporelle de la particule.

\section{Méthode dynamique - Paquet d'ondes gaussien}

\[\Psi(x,t) = \int A(k) e^{i(kx-\omega(k)t)} dk \quad \text{avec} \quad \omega(k) = \frac{1}{2m}\hbar k^2\]

La particule est représentée par un paquet d'ondes localisé, c'est à dire une superposition d'ondes planes proche de $k_0$. Elle a donc une énergie moyenne, mais pas parfaitement définie (ondes stationnaires).

Cette méthode permet de suivre l'évolution du paquet dans le temps notamment lorsqu'il rencontre un potentiel.
Après l'interaction, le paquet se divise en une partie transmise et une partie réfléchie.

Cette approche est plus réaliste, car elle décrit une particule en mouvement et permet d'observer des effets comme les interférences.

\end{document}